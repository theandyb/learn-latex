\documentclass[11pt]{article}
\usepackage[margin=1in]{geometry}
\usepackage{verbatim} % required for \begin{comment}
\usepackage{syntonly}
\usepackage{amsmath}
\usepackage{lipsum}
%\syntaxonly %uncomment this line to suppress pdf output

\hyphenation{FORTRAN Hy-phen-a-tion} %allows hyphenation to by hyphenated and explicitly prevents FORTRAN from being hyphenated

\author{A. Beck}
\title{Words, etc}

\begin{document}

\maketitle

\begin{abstract}
This is a document about nothing.
\end{abstract}

\tableofcontents

\section{The Beginning}
Experienced \TeX{} users are \TeX perts, and know how to use whitespace. %note difference when we do and do not use {} with \TeX command

Please, start a new line right here!\newline
Thank you!

This is another
\begin{comment}
rather stupid,
but helpful
\end{comment}
example for embedding
comments in your document.

\newpage
This text should appear on a new page.

\section{Another section coming at ya}
The parameter \mbox{\emph{filename}} should contain the name of the file. %mbox forces its contents to be on the same line (i.e. prevents hyphenated split if at end of line)

\subsection{What's in the box?}
\label{box}\fbox{Is there a box around this line of text???}

\subsection{Quotation woes}
``\LaTeX{} does not make things easy to remember in regards to quotation marks,'' moaned the whiny graduate student.

\subsection{Dashes and other symbols}
\subsubsection{HOW DASHING}
daughter-in-law, X-rated\\ %hyphen
pages 13--67\\ %en-dash
yes---or no? \\ %em-dash
$0$, $1$ and $-1$ %minus-sign

\subsubsection{Some more symbolz}
There is a tilde command (\~{}), but it might be more visually appealing to use $\sim$

It's $-30\,^{\circ}\mathrm{C}$.
I will soon start to
super-conduct.

Not like this ... but like this:\\
New York, Tokyo, Budapest, \ldots

\subsubsection{Spice things up with some accents, etc}

H\^otel, na\"\i ve, \'el\`eve,\\
sm\o rrebr\o d, !`Se\~norita!,\\
Sch\"onbrunner Schlo\ss{}
Stra\ss e

\subsection{Referring to the box above}
Do you remember the box above on page \pageref{box}? It was in section \ref{box}. You don't? That's a pity.\footnote{It's not too far up this very same page\ldots}

\section{Environments}
\subsection{Itemize, Enumerate, and Decription}

\flushleft
\begin{enumerate}
\item You can nest the list
environments to your taste:
\begin{itemize}
\item But it might start to
look silly.
\item[-] With a dash.
\end{itemize}
\item Therefore remember:
\begin{description}
\item[Stupid] things will not
become smart because they are
in a list.
\item[Smart] things, though,
can be presented beautifully
in a list.
\end{description}
\end{enumerate}

\subsection{Quote, Quotation}

One might be interested in using the quote environment:
\begin{quote}
If what one is quoting is rather short
\end{quote}

The quotation environment is useful for longer quotes that span multiple paragraphs:
\begin{quotation}
\lipsum[1-4]
\end{quotation}

\subsection{Other environments}
\begin{verbatim*}
Notice how all of 
the spaces are emphasized?
\end{verbatim*}

\subsubsection{Tables?}
\begin{tabular}{|r|l|}
\hline
7C0 & hexadecimal \\
3700 & octal \\ \cline{2-2}
11111000000 & binary \\
\hline \hline
1984 & decimal \\
\hline
\end{tabular}

What follows next is a table of my own concoction:

\begin{tabular}{|c|c|c|}
\hline
Andy & Is & Cool \\ \cline{1-3}
Don't & You & Know? \\
No & cline & here \\
\hline
What & happens & Now? \\
\hline
\end{tabular}

This is a table where we've removed the leading space:

\begin{tabular}{@{} l @{}}
\hline
no leading space\\
\hline
\end{tabular}

When actually creating tables, you'll likely want to use the booktabs package.

\end{document}

